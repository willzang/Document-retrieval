%运行核心:texlive2018
%program run:xeLatex(运行run.bat即可)
%%%%%%%%%%%%%%%%%%%%%%%%%%%%%%%%%%%%%%%

\documentclass[9pt,UTF8]{beamer}
%\usepackage{ctex}
\usepackage{xeCJK}%加载中文字体宏包,使用了ctex,就不能使用下面的setCJKmainfont了。
%\setCJKmainfont{微软雅黑}%微软雅黑字体,但不是所有电脑都能使用
%\setCJKmainfont{宋体}
\setCJKmainfont[Path=fonts/]{msyh.ttc}%导入本地fonts文件夹下的字体“微软雅黑”。P.S. simsun.ttc为宋体,msyhbd.ttc为微软雅黑粗体,msyhl.ttc为微软雅黑细体
\beamertemplateshadingbackground{white}{blue!10}%设置灰白渐变背景
%\beamertemplategridbackground[.25cm] %设置网格背景
\usepackage{indentfirst} %设置首段首行缩进
\setlength{\parindent}{2em} %设置缩进 2 个字符宽度
%\linespread{1.245}%1.5倍行距,加载后,会无法显示脚注

\usetheme{Madrid}%主题,蓝色主题,并在每页最下栏囊括标题内容

\usepackage{multirow}

\newcommand{\blue}{\textcolor[rgb]{0.20,0.20,0.70}}%定义blue为蓝色字体
\newcommand{\BLUE}[1]{\textbf{\blue{\large #1}}}%BLUE为放大、加粗的蓝色字体
\newcommand{\jie}[1]{\textbf{\blue{\LARGE #1}}}%"jie(节,PPT叶中的小节)"为最大加粗的蓝色字体

%\tiny \scriptsize \footnotesize \small \normalsize \large \Large \LARGE \huge \Huge.%字体大小:

\usepackage{graphicx}%图片宏包
\setbeamertemplate{caption}[numbered]%beamer 自身默认图表是没有编号的,如果需要把编号调出来,只需要在导言区加上此段
%\usepackage{setspace}%使用间距宏包

\graphicspath{{pictures/}}   % 设置图片所存放的目录

\usepackage{tcolorbox}%文字方框宏包

\usepackage{amssymb}%符号宏包

%%%%%%%%%%%%%标题%%%%%%%%%%%
\title{高血压药物介绍}%选自《2018中国高血压防治指南修订版》
\author{张旭栋}
\institute[Hangzhou Ding Qiao Hospital]{杭州市丁桥医院\\
\textit{E-mail: zang\_xd@outlook.com}
}
\date{2018.12.20}

%%%%%%%%%%%%%正文%%%%%%%%%
\begin{document}
\include{./body}
\end{document}%运行核心:texlive2018
%program run:xeLatex(运行run.bat即可)
%%%%%%%%%%%%%%%%%%%%%%%%%%%%%%%%%%%%%%%

\documentclass[9pt,UTF8]{beamer}
%\usepackage{ctex}
\usepackage{xeCJK}%加载中文字体宏包,使用了ctex,就不能使用下面的setCJKmainfont了。
%\setCJKmainfont{微软雅黑}%微软雅黑字体,但不是所有电脑都能使用
%\setCJKmainfont{宋体}
\setCJKmainfont[Path=fonts/]{msyh.ttc}%导入本地fonts文件夹下的字体“微软雅黑”。P.S. simsun.ttc为宋体,msyhbd.ttc为微软雅黑粗体,msyhl.ttc为微软雅黑细体
\beamertemplateshadingbackground{white}{blue!10}%设置灰白渐变背景
%\beamertemplategridbackground[.25cm] %设置网格背景
\usepackage{indentfirst} %设置首段首行缩进
\setlength{\parindent}{2em} %设置缩进 2 个字符宽度
%\linespread{1.245}%1.5倍行距,加载后,会无法显示脚注

\usetheme{Madrid}%主题,蓝色主题,并在每页最下栏囊括标题内容

\usepackage{multirow}

\newcommand{\blue}{\textcolor[rgb]{0.20,0.20,0.70}}%定义blue为蓝色字体
\newcommand{\BLUE}[1]{\textbf{\blue{\large #1}}}%BLUE为放大、加粗的蓝色字体
\newcommand{\jie}[1]{\textbf{\blue{\LARGE #1}}}%"jie(节,PPT叶中的小节)"为最大加粗的蓝色字体

%\tiny \scriptsize \footnotesize \small \normalsize \large \Large \LARGE \huge \Huge.%字体大小:

\usepackage{graphicx}%图片宏包
\setbeamertemplate{caption}[numbered]%beamer 自身默认图表是没有编号的,如果需要把编号调出来,只需要在导言区加上此段
%\usepackage{setspace}%使用间距宏包

\graphicspath{{pictures/}}   % 设置图片所存放的目录

\usepackage{tcolorbox}%文字方框宏包

\usepackage{amssymb}%符号宏包

%%%%%%%%%%%%%标题%%%%%%%%%%%
\title{高血压药物介绍}%选自《2018中国高血压防治指南修订版》
\author{张旭栋}
\institute[Hangzhou Ding Qiao Hospital]{杭州市丁桥医院\\
\textit{E-mail: zang\_xd@outlook.com}
}
\date{2018.12.20}

%%%%%%%%%%%%%正文%%%%%%%%%
\begin{document}
\include{./body}
\end{document}%运行核心:texlive2018
%program run:xeLatex(运行run.bat即可)
%%%%%%%%%%%%%%%%%%%%%%%%%%%%%%%%%%%%%%%

\documentclass[9pt,UTF8]{beamer}
%\usepackage{ctex}
\usepackage{xeCJK}%加载中文字体宏包,使用了ctex,就不能使用下面的setCJKmainfont了。
%\setCJKmainfont{微软雅黑}%微软雅黑字体,但不是所有电脑都能使用
%\setCJKmainfont{宋体}
\setCJKmainfont[Path=fonts/]{msyh.ttc}%导入本地fonts文件夹下的字体“微软雅黑”。P.S. simsun.ttc为宋体,msyhbd.ttc为微软雅黑粗体,msyhl.ttc为微软雅黑细体
\beamertemplateshadingbackground{white}{blue!10}%设置灰白渐变背景
%\beamertemplategridbackground[.25cm] %设置网格背景
\usepackage{indentfirst} %设置首段首行缩进
\setlength{\parindent}{2em} %设置缩进 2 个字符宽度
%\linespread{1.245}%1.5倍行距,加载后,会无法显示脚注

\usetheme{Madrid}%主题,蓝色主题,并在每页最下栏囊括标题内容

\usepackage{multirow}

\newcommand{\blue}{\textcolor[rgb]{0.20,0.20,0.70}}%定义blue为蓝色字体
\newcommand{\BLUE}[1]{\textbf{\blue{\large #1}}}%BLUE为放大、加粗的蓝色字体
\newcommand{\jie}[1]{\textbf{\blue{\LARGE #1}}}%"jie(节,PPT叶中的小节)"为最大加粗的蓝色字体

%\tiny \scriptsize \footnotesize \small \normalsize \large \Large \LARGE \huge \Huge.%字体大小:

\usepackage{graphicx}%图片宏包
\setbeamertemplate{caption}[numbered]%beamer 自身默认图表是没有编号的,如果需要把编号调出来,只需要在导言区加上此段
%\usepackage{setspace}%使用间距宏包

\graphicspath{{pictures/}}   % 设置图片所存放的目录

\usepackage{tcolorbox}%文字方框宏包

\usepackage{amssymb}%符号宏包

%%%%%%%%%%%%%标题%%%%%%%%%%%
\title{文检}%选自《2018中国高血压防治指南修订版》
\author{张旭栋}
\institute[Hangzhou Ding Qiao Hospital]{杭州市丁桥医院\\
\textit{E-mail: zang\_xd@outlook.com}
}
\date{2018.12.20}

%%%%%%%%%%%%%正文%%%%%%%%%
\begin{document}
\include{./body}
\end{document}
